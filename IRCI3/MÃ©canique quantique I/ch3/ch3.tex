\chapter{Représentations position - impulsion}
Commençons par rappeler quelques notions dans la base position
\begin{equation}
\ket{\psi} = \int d\vec r\ \psi(\vec{r})\ket{\vec{r}}
\end{equation}
où les coefficients de Fourier sont donnés par $\psi(\vec{r}) = \bra{\vec{r}}\ket{\psi}$.
\begin{equation}
\bra{\psi}\ket{\psi} = \iint d\vec r\ d\vec r'\psi^*(\vec{r'})\psi(\vec{r})\underbrace{
\bra{\vec{r'}}\ket{\vec{r}}}_{\delta(\vec{r},\vec{r'})} = \int d\vec r |\psi(\vec{r}|^2=1
\end{equation}
Avec la décomposition spectrale $\hat{\vec{r}} = \int d\vec{r}\ \vec{r}\ket{\vec{r}}\bra{\vec{r}}$, 
on peut écrire\footnote{Si tu veux je peux détailler le raisonnement pour la valeur moyenne...}
\begin{equation}
\mathbb{P}(\vec{r}) = \underbrace{\bra{\psi}\ket{\vec{r}}}_{\psi^*(\vec{r})}\underbrace{
\bra{\vec{r}}\ket{\psi}}_{\psi(\vec{r})} = |\psi(\vec{r})|^2,\qquad\qquad\langle\hat{\vec{r}}\rangle = 
\bra{\psi}\hat{\vec{r}}\ket{\psi} = \int d\vec{r}\ \vec{r}\underbrace{\bra{\psi}\ket{\vec{r}}
\bra{\vec{r}}\ket{\psi}}_{|\psi(\vec{r})|^2}
\end{equation}

\section{Opérateur impulsion}
Pour motiver physiquement l'opérateur impulsion, repartons de la notion d'onde de de Broglie que 
l'on peut associer à une particule libre ayant une certaine vitesse. 
\subsection{Onde de De Broglie, paquets d'onde}
		\subsubsection{Onde de De Broglie}
		Grâce à de Broglie, on associe une onde à toute particule en mouvement.
		Par analogie avec les concepts de l'électromagnétisme, décrivons cette onde au moyen d'exponentielles complexes.
		%On peut voir une particule comme une onde, de Broglie a réussi à donner les "paramètres" de cette onde.
		% Voyons ça comme une onde, par analogie à l'électromagnétisme
		\begin{equation}
		\psi(\vec{r},t) = \psi_0e^{i(\vec{k}\vec{r}-\omega t)}
		\end{equation}
		De Broglie pose les valeurs qu'il faut associer à $\vec k$ et $\omega$ pour une particule ayant 
		une certaine vitesse et énergie. Le point de départ est
		\begin{equation}
		\lambda = \dfrac{h}{p},\qquad k=\dfrac{2\pi}{\lambda},\qquad \vec k=\dfrac{\vec{p}}{\hbar}
		\end{equation}
		Pour la fréquence, par analogie à l'optique\footnote{As-tu vérifié la chronologie ?}
		\begin{equation}
		\nu = \dfrac{E}{h},\qquad \omega = 2\pi\nu = \dfrac{E}{\hbar}
		\end{equation}
		En substituant ces expressions dans l'expression de l'onde plane
		\begin{equation}
		\psi(\vec{r},t) = \psi_0\ e^{\frac{i}{\hbar}\left(\vec{p}.\vec{r}-Et\right)}\qquad \text{ où }\ \psi_0 =
		\dfrac{1}{(2\pi\hbar)^{3/2}} \text{ pour la normalisation}
		\end{equation}
		Cette solution étant écrite par analogie avec l'EM, il est possible d'écrire une équation 
		d'onde que cette onde va vérifier. Regardons par exemple si l'on effectue
		\begin{equation}
		\begin{array}{lll}
		\bullet i\hbar\frac{\partial}{\partial t}\psi &= i\hbar \psi_0\frac{E}{i\hbar}e^{\dots} &=
		E\psi(\vec{r},t)\\
		\bullet \Delta \psi(\vec{r},t) &= \psi_0\dfrac{p_x^2+p_y^2+p_z^2}{(-i\hbar)^2}e^{\dots} &= -
		\dfrac{p^2}{\hbar^2}\psi(\vec{r},t)
		\end{array}
		\end{equation}
		On en tire pour une particule libre
		\begin{equation}
		-\dfrac{\hbar^2}{2m}\Delta \psi(\vec{r},t) = \dfrac{p^2}{2m}\psi(\vec{r},t) = E\psi(\vec{r},t)
		\end{equation}
		On retrouve le classique
		\begin{equation}
		\underline{i\hbar\dfrac{\partial}{\partial t}\psi(\vec{r},t)= -\dfrac{\hbar^2}{2m}\Delta\psi(\vec{r},t)}
		\end{equation}
		Il s'agit d'une ED de type d'onde qui est bien vérifiée par l'onde plane construite ci-dessus. 
		Cette ED correspond à une particule libre. Elle est linéaire : toute combili de solution est 
		solution. Au lieu de prendre une onde de de Broglie monocinétique correspondant à une onde 
		monochromatique, on peut considérer un paquet d'onde. Par exemple
		\begin{equation}
		\left.\begin{array}{ll}
		p_1 \rightarrow \psi_1\\
		p_2 \rightarrow \psi_2
		\end{array}\right\}\longrightarrow \alpha\psi_1+\beta\psi_2
		\end{equation}
		Un paquet d'onde est une superposition continue de particules avec toute une gamme de vitesse 
		possible. Il y correspondra une onde de de Broglie nommée \textit{paquet d'ondes}.

		\subsubsection{Paquet d'ondes}
		En substituant le préfacteur $\psi_0$, on obtient
		\begin{equation}
		\psi(\vec{r},t) = \dfrac{1}{(2\pi\hbar)^{3/2}}\int d\vec{p}\ \phi(\vec{p})e^{\frac{i}{\hbar}(\vec{p}
		\vec{r}-Et)}
		\end{equation}
		
	\subsection{Dualité des bases position et impulsion (transformée de Fourier)}
	La même expression, en notation de Dirac :
	\begin{equation}
	\ket{\psi} = \int d\vec{p}\ \phi(\vec{p})\ket{\vec{p}}
	\end{equation}
	Il s'agit d'un mélange (avec un certain poids) de kets $\bra{\vec{p}}$. En refermant
		\begin{equation}
	\psi(\vec{r},t) = \bra{\vec{r}}\ket{\psi} = \int d\vec{p}\ \phi(\vec{p})\underbrace{
	\bra{\vec{r}}\ket{\vec{p}}}_{(*)}
	\end{equation}
	où $\DS (*) = \frac{1}{(2\pi\hbar)^{3/2}}e^{\frac{i}{\hbar}(\vec{p}\vec{r}-Et)}$.
	Pour une raison d'élégance, intégrons le facteur de phase temporel dans un ket.	
%	 On voudrait 
%	faire la même chose mais en intégrant la phase dans le ket
	\begin{equation}
		\psi(\vec{r},t) = \frac{1}{(2\pi\hbar)^{3/2}}\int d\vec{p} \underbrace{\phi(\vec{p})e^{-\frac{i}
	{\hbar}Et}}_{\phi(\vec{p},t)}e^{\frac{i}{\hbar}\vec{p}.\vec{r}}
	\end{equation}
	\textbf{Attention} : ce n'est plus le même $\ket{\vec{p}}$ qu'à la précédente expression, la 
	définition à ici changée afin de pouvoir écrire
	\begin{equation}
	\ket{\psi} = \int d\vec{p}\ \phi(\vec{p},t)\ket{p}
		\end{equation}
	En refermant
	\begin{equation}
	\psi(\vec{r},t) = \bra{\vec{r}}\ket{\psi} = \int d\vec{p}\ \phi(\vec{p},t)
	\underbrace{\bra{\vec{r}}\ket{\vec{p}}}_{(**)}
		\end{equation}
	où $\DS (**)= \frac{1}{(2\pi\hbar)^{3/2}}e^{\frac{i}{\hbar}\vec{p}.\vec{r}}$ est le 
	\textbf{noyau de la transformée de Fourier}.\\
	
	Résumons
	\begin{equation}
		\ket{\psi}\left\{\begin{array}{ll}
	= \int d\vec{r}\ \psi(\vec{r},t)\ket{\vec{r}} & \rightarrow\text{Base pos.}\\
	= \int d\vec{p}\ \phi(\vec{p},t)\ket{\vec{p}} & \rightarrow\text{Base imp.}		
	\end{array}\right.
	\end{equation}
	avec
		\begin{equation}
	\underline{\bra{\vec{r}}\ket{\vec{p}} = \frac{1}{(2\pi\hbar)^{3/2}}e^{\frac{i}{\hbar}\vec{p}.\vec{r}}}
	\end{equation}
	Il s'agit de la \textit{formule de changement de base}. Elle est particulièrement intéressante :
	\begin{equation}
	\begin{array}{ll}
		\psi(\vec{r},t) = \bra{\vec{r}}\ket{\psi} &= \int d\vec{p}\ \phi(\vec{p},t)\ \bra{\vec{r}}\ket{\vec{p}}\\
	&= \frac{1}{(2\pi\hbar)^{3/2}}\int d\vec{p}\ \phi(\vec{p},t)e^{\frac{i}{\hbar}\vec{p}\vec{r}}\\
	\phi(\vec{p},t) = \bra{\vec{p}}\ket{\psi} &= \int d\vec{r}\ \psi(\vec{r},t)\ \bra{\vec{p}}\ket{\vec{r}}\\
	&= \frac{1}{(2\pi\hbar)^{3/2}}\int d\vec{r}\ \psi(\vec{r},t)e^{-\frac{i}{\hbar}\vec{p}\vec{r}}		
	\end{array}
		\end{equation}
	On remarque qu'il existe une transformée de Fourier qui lie les bases position et impulsion.
	On peut écrire
	\begin{equation}
	\underline{\phi(\vec p) = TF[\psi(\vec{r})]},\qquad\qquad \underline{\psi(\vec r) = TF^{-1}
	[\phi(\vec{p})]}
	\end{equation}
		
	
	Il en découle une série de propriétés
	\begin{itemize}
	\item[i.] \textit{Théorème de Parseval }: le produit scalaire de deux fonctions vaut le produit 
	scalaire des TF de ces deux fonctions. 
		\begin{equation}
	\int f_1(\vec{r})f_2^*(\vec{r}) d\vec{r} = \int F_1(\vec{p})F_2^*(\vec{p})d\vec{p}
	\end{equation}
	Dans notre cas :
	\begin{equation}
	\bra{\psi_1}\ket{\psi_2} \left\{\begin{array}{ll}
		= \int d\vec{r}\ \underbrace{\bra{\psi_1}\ket{\vec{r}}}_{\psi_1^*(\vec{r})}
	\underbrace{\bra{\vec{r}}\ket{\psi_2}}_{\psi_2(\vec{r})}\\
	= \int d\vec{p}\ \underbrace{\bra{\psi_1}\ket{\vec{p}}}_{\phi_1^*(\vec{p})}
	\underbrace{\bra{\vec{p}}\ket{\psi_2}}_{\phi_2(\vec{p})}		
	\end{array}\right.
	\end{equation}
		La normalisation est conservée par le changement de base
	\begin{equation}
	1 = \bra{\psi}\ket{\psi} = \int d\vec{r}\ |\psi(\vec{r})|^2 = \int d\vec{p}\ |\phi(\vec{p})|^2
	\end{equation}
	
	\item[ii.] \textit{Relation d'incertitude $\hat{x}-\hat{p}$}: on peut voir Robertson comme une 
	conséquence de ces TF : la TF d'une fonction étroite sera large et inversément. Il y a quelque chose qui peut
	 s'apparenter à une relation d'incertitude $\hat x-\hat p$. Nous avons
	 \begin{equation}
	 \langle p^2\rangle = \int |\phi(p)|^2\ p^2\ dp
	 \end{equation}
	Alors
	\begin{equation}
	\begin{array}{ll}
	\Delta p^2 &= \langle p^2 \rangle-\langle p\rangle^2 \longleftarrow |\phi(p)|^2\\
	\Delta x^2 &= \langle x^2 \rangle-\langle x\rangle^2 \longleftarrow |\psi(x)|^2
	\end{array}\Longrightarrow \Delta \hat x\Delta\hat{p} \geq\frac{\hbar}{2}
	\end{equation}
	Si on a une paire de fonctions (ici $\psi$ et $\phi$) qui sont "connectées" par une TF, dans la 
	théorie des TF le produit des variances peut être minoré par une constante. Cette démonstration
	ne sera pas faite ici mais cela fournit un autre point de vue sur les relations d'incertitude.
		
	\item[iii.] \textit{Dérivée }: la dérivée d'une fonction, au niveau de sa TF, est une multiplication 
	par $i*$(variable conjugée). Ceci permet de définir proprement l'opérateur impulsion en base 
	position.
	\begin{equation}
	\psi(\vec{r}) = \frac{1}{(2\pi\hbar)^{3/2}}\int\phi(p)e^{i\frac{\vec{p}.\vec{r}}{\hbar}}\ d\vec{p}\qquad
	\text{ où }\ \vec{r} = (x_1,x_2,x_3)
	\end{equation}
	En dérivant
	\begin{equation}
	\frac{\partial}{\partial x_j} \psi(\vec{r}) = \frac{1}{(2\pi\hbar)^{3/2}}\int\underline{\phi(p)\frac{i}
	{\hbar} p_j}	e^{i\frac{\vec{p}.\vec{r}}{\hbar}}\ d\vec{p}
	\end{equation}
	Ou encore (avec un peu de fainéantise)
	\begin{equation}
	-i\hbar\dfrac{\partial}{\partial x_j} \psi(\vec{r}) = \frac{1}{\dots}\int \underline{\phi(p)p_j}\dots
	\end{equation}
	ce qui permet de définir l'opérateur impulsion.
	
	\end{itemize}

	\subsection{Opérateur impulsion comme un opérateur différentiel en base position}
	La dernière propriété montre que si on dérive dans le domaine position, 
	cela revient à multiplier par $\frac{ip_j}{\hbar}$.  Une façon simple est d'écrire la valeur moyenne de 	
	l'impulsion
	\begin{equation}
	\begin{array}{ll}
	\langle p_j\rangle = \bra{\psi}p_j\ket{\psi} &= \int d\vec{p}\ \phi^*(\vec{p})\underline{p_j\phi(\vec{p})}\\
	&= \int d\vec{r}\ \psi^*(\vec{r})(-i\hbar)\frac{\partial}{\partial x_j}\psi(\vec{r})
	\end{array}
	\end{equation}	
	La première ligne n'est que la ré-écriture dans la base impulsion. Pour passer à la seconde ligne, 
	on utilise Parseval sur le terme souligné.\footnote{Ce passage n'est pas très clair :(}
	 Or, nous avons également
	\begin{equation}
	\langle p_j\rangle = \int d\vec{r}	 \bra{\psi}\ket{\vec{r}}\bra{\vec{r}}\hat{P_j}\ket{\psi}
	\end{equation}
	Et donc
	\begin{equation}
	\underline{\bra{\vec{r}}\hat{P_j}\ket{\psi} = -i\hbar\dfrac{\partial}{\partial x_j}\bra{\vec{r}}\ket{\psi}}
	\end{equation}
	En généralisant avec $\hat{\vec{p}}$, un opérateur vectoriel, on obtient la définition en
	base position de l'opérateur impulsion
	\begin{equation}
	\underline{\bra{\vec{r}}\hat{\vec{p}}\ket{\psi} = -i\hbar\vec{\nabla_r}\bra{\vec{r}}\ket{\psi}}
	\end{equation}
	Pour s'amuser, que vaut $[\hat{r_j},\hat{p_k}]$ ? 
	\begin{equation}
	\begin{array}{ll}
	\bra{\vec{r}}[\hat{r_j},\hat{p_k}]\ket{\psi} &= \bra{\vec{r}}\hat{r_j}\hat{p_k}-\hat{p_k}\hat{r_j}\ket{\psi}\\
	&= r_j\bra{\vec{r}}\hat{p_k}\ket{\psi}-\bra{\vec{r}}\hat{p_k}\left(\hat{r_j}\ket{\psi}\right)\\
	&= \hat{r_j}(-i\hbar)\frac{\partial}{\partial x_k}\bra{\vec{r}}\ket{\psi}-(-i\hbar)\frac{\partial}{\partial
	 x_k}\underbrace{\left(\bra{\vec{r}}\hat{r_j}\ket{\psi}\right)}_{x_j\bra{\hat{\vec{r}}}\ket{\psi}}\\

	&= i\hbar\frac{\partial x_j}{\partial x_k}\bra{\vec{r}}\ket{\psi}\\
	&= i\hbar\delta_{jk}\bra{\vec{r}}\ket{\psi}\qquad\qquad\forall \psi,\vec{r}
	\end{array}
	\end{equation}
	Dès lors
	\begin{equation}
	[\hat{r_j},\hat{p_k}]\ket{\psi} = i\hbar\delta_{jk}\ket{\psi}
	\end{equation}
	Le commutateur vaut alors
	\begin{equation}
	[\hat{r_j},\hat{p_k}] = i\hbar\delta_{jk}
	\end{equation}
	
	

\section{Équation de Schrödinger en base position (mécanique ondulatoire)}
Commençons par quelques rappels (sans les $\hat{\ }$)
\begin{equation}
\bra{x}p\ket{\psi} = -i\hbar \frac{\partial}{\partial x}\bra{x}\ket{\psi}
\end{equation}
où\footnote{facteur $1/2$ à la place de $3/2$ car l'exemple est à une dimension}
\begin{equation}
\bra{x}\ket{\psi} = \int dp\ \bra{x}\ket{p}\bra{p}\ket{\psi} = \frac{1}{(2\pi\hbar)^{1/2}}
\int dp\ e^{\frac{i}{\hbar}x.p}\bra{p}\ket{\psi}
\end{equation}
En considérant la dérivée
\begin{equation}
\frac{\partial}{\partial x}\bra{x}\ket{\psi} = \frac{1}{(2\pi\hbar)^{3/2}}\int dp\ \frac{i}{\hbar}p
e^{\frac{i}{\hbar}x.p}\bra{p}\ket{\psi}
\end{equation}
Nous avions alors obtenu
\begin{equation}
-i\hbar\frac{\partial}{\partial x}\bra{x}\ket{\psi} = \int dp\ \bra{x}\ket{p}p\bra{p}\ket{\psi} = 
\bra{x}\hat{p}\ket{x}
\end{equation}
Ce qui donne en 3D
\begin{equation}
\bra{\vec{r}}\hat{\vec{p}}\ket{\psi} = -i\hbar\vec{\nabla_r}\bra{\vec{r}}\ket{\psi}
\end{equation}
Pour vérifier que cet opérateur impulsion est hermitien dans la base position (on sait qu'il
l'est déjà dans la base impulsion), on voudrait montrer qu'il est égal à son 
adjoint. Pour se faire, on prend n'importe quel élément de matrice et on regarde s'il est égal avec
l'élément de matrice de son adjoint
\begin{equation}
\forall\psi,\ \forall \phi : \bra{\psi}\hat{p}\ket{\phi} ?= \bra{\phi}\hat{p}\ket{\psi}^*
\end{equation}
Effectuons de part et d'autre cette égalité à vérifier
\begin{equation}
\begin{array}{ll}
\DS \int dx\ \psi^*(x)(-i\hbar)\frac{\partial}{\partial x}\phi(x) &?= \left\{\DS \int dx\ \phi^*(x)(-i\hbar)
\frac{\partial}{\partial x}\psi(x)\right\}^*\\
&?=\DS \int dx\ \phi(x)(i\hbar)\frac{\partial}{\partial x}\psi^*(x)
\end{array}
\end{equation}
Si cette égalité est vraie, la différence doit être nulle
\begin{equation}
\int dx\ \left[\DS \phi(x)\frac{\partial}{\partial x}\psi^*(x) + \psi^*(x)\frac{\partial}{\partial x}\phi(x)
\right]?=0
\end{equation}
Il s'agit de l'expression de la dérivée d'un produit
\begin{equation}
\int dx\ \dfrac{\partial}{\partial x}\left[\phi(x)\psi^*(x)\right] = \left[\phi(x)\psi^*(x)\right]_{-\infty}^{
+\infty} = 0
\end{equation}
Les fonctions d'onde étant de carrés sommables, la dernière égalité est vérifiée.
On a bien redémontré que l'opérateur impulsion est hermitien en base position. \\

En base impulsion $\left\{\ket{p}\right\}$, on a 
\begin{equation}
\bra{p}\hat{x}\ket{\psi} = i\hbar \frac{\partial}{\partial p}\bra{p}\ket{\psi}
\end{equation}
Ou en 3D
\begin{equation}
\bra{p}\hat{\vec{r}}\ket{\psi} = i\hbar\vec{\nabla_p}\bra{\vec{p}}\ket{\psi}
\end{equation}
On peut refaire exactement le même genre de calcul pour arriver au mêmes conclusions\footnote{Savoir 
montrer que $\hat{x}$ est hermitien dans la base impulsion.} et obtenir une analogique parfaite.\\

Que se passe-t-il quand on plonge l'équation de Schrödinger dans une base ou l'autre ?

	\subsection{Équation de Schrödinger en base position}
	En notation de Dirac
	\begin{equation}
	i\hbar\frac{\partial}{\partial t}\ket{\psi(t)} = \hat{H}\ket{\psi(t)}
	\label{eq:6.5}
	\end{equation}
	Dans un espace à trois dimensions, pour une particule plongée dans un potentiel $V(\vec r)$, l'hamiltonien 
	s'écrit
	\begin{equation}
	\hat{H} = \underbrace{\frac{p^2}{2m}}_{\hat{K}} + \underbrace{V(\vec{r})}_{\hat{V}}
	\end{equation}
	Si on referme par un bra, on obtient la fonction d'onde
	\begin{equation}
\underline{i\hbar \frac{\partial}{\partial t}\psi(\vec{r},t) = \overbrace{-\frac{\hbar}
	{2m}\Delta_r\psi(\vec{r},t)}^{\text{En. cin.}} + \overbrace{V(\vec{r})\psi(\vec{r},t)}^{\text{En. pot.}}}
	\end{equation}
	Dans le cas où le potentiel est nul, on retombe sur la propagation d'ondes libres.
	%il s'agit bien de l'ED d'une particule libre donnant comme solution les ondes de de Broglie. 
	On peut réécrire la même chose de façon un peu plus rigoureuse. En partant de l'ED de Schrödinger 
	générale dépendante du temps \eqref{eq:6.5}, on peut écrire
	\begin{equation}
	\forall\vec{r}\ :\ i\hbar\frac{\partial}{\partial t}\underbrace{\bra{\vec{r}}\ket{\psi}}_{\psi(\vec{r},
	t)} = \bra{\vec{r}}\hat{H}\ket{\psi(\vec{r},t)}
	\end{equation}
	Un élément de matrice entre braket donne la fonction d'onde. Regardons terme à terme
	\begin{enumerate}
	\item \textit{Énergie cinétique}. Nous avons, en partant de la définition, en appliquant une 
	seconde fois la définition et en passant en 3D
	\begin{equation}
	\begin{array}{ll}
	\bra{\vec{r}}\hat{\hat{p_x}}\ket{\psi} &= -i\hbar\frac{\partial}{\partial x}\bra{\vec{r}}\ket{\psi}\\
	\bra{\vec{r}}\hat{p_x^2}\ket{\psi} &= -\hbar^2\frac{\partial^2}{\partial x^2}\bra{\vec{r}}\ket{\psi}	\\
	\bra{\vec{r}}\hat{K}\ket{\psi} &= \frac{-\hbar^2}
	{2m}\Delta_r\underbrace{\bra{\vec{r}}\ket{\psi}}_{\psi(\vec{r},t)}
	\end{array}
	\end{equation}
	
	\item \textit{Potentiel local}. $\hat V$ diagonalisable en base position
	\begin{equation}
	\bra{\vec{r}}\hat{V}\ket{\psi} = V(\vec{r})\bra{\vec{r}}\ket{\psi}
	\end{equation}
	
	\item \textit{Potentiel non local}. Parfois le potentiel est non local \footnote{Ce genre de potentiels se retrouve
	dans l'étude de systèmes à particules identiques} (pas diagonalisable en base position) : on vient 
	alors placer la relation de fermeture en $\vec r$ et on intègre sur $\vec{r'}$.
	\begin{equation}
	\int d\vec{r'}\ \underbrace{\bra{\vec{r}}\hat{V}\ket{\vec{r'}}}_{V(\vec{r},\vec{r'})}\underbrace{
	\bra{\vec{r'}\ket{\psi}}}_{\psi(\vec{r'})}
	\end{equation}
	\end{enumerate}
	Ce développement terme à terme donne une ED de Schrödinger pour un potentiel non-local
	\begin{equation}
	\underline{i\hbar\frac{\partial}{\partial t}\psi(\vec{r},t) = -\frac{\hbar^2}{2m}\Delta_r\psi(\vec{r},t)+\int 
	d\vec{r'}\ V(\vec{r},\vec{r'})\psi(\vec{r'},t)}
	\end{equation}
	Dès lors, la fonction d'onde en un point ne dépend plus uniquement du potentiel en ce point mais
	également de la valeur de la fonction d'onde et du potentiel en tous les autres points.

	\subsection{Équation de Schrödinger indépendante du temps (états stationnaires)}
	Si le système est isolé ($\hat H$ indépendant de $t$, pas de couplage avec l'environnement) on peut commencer 
	par la résolution de l'ED de Schrödinger indépendante du temps (qui est une équation aux valeurs propres)
	 que l'on peut écrire en base position
	\begin{equation}
	-\frac{\hbar^2}{2m}\Delta_r\psi(\vec{r}) + V(\vec{r})\psi(\vec{r}) = E\psi(\vec{r})\qquad\longrightarrow
	\qquad \left\{E_n,\psi_n(\vec{r})\right\}
	\end{equation}
	Ceci forme les états stationnaires car si
	\begin{equation}
	\psi(\vec{r},t) = \psi_n(\vec{r})e^{-\frac{i}{\hbar}E_nt}
	\end{equation}
	on a un état stationnaire de l'équation dépendant du temps. C'est bien stationnaire car on retrouve 
	$\psi_n(\vec{r})$ à une phase globale près. Si l'on s'intéresse aux probabilités
	\begin{equation}
	|\psi(\vec{r},t)|^2 = |\psi_n(\vec{r})|^2
	\end{equation}
	ce qui est bien indépendant du temps. \\
	
	Une combili de ces ED stationnaires sera toujours solutions, mais cette fois-ci elle évoluera dans le 
	temps. La forme générale de cette solution peut s'écrire
	\begin{equation}
	\psi(\vec{r},t) = \sum_n c_n\psi_n(\vec{r})e^{-\frac{i}{\hbar}E_nt}
	\end{equation}
	Ici comme on somme les phases, les systèmes vont évoluer dans le temps.
	
\section{Résolution de quelques cas simples (états liés)}
Parlons de quelques exemples dans le cas particulier de l'ED de Schrödinger en base impulsion. Il y a 
deux grand type de problèmes
\begin{enumerate}
\item Particule piégée dans un potentiel %, classiquement piégée
\item Particule avec une énergie cinétique suffisante pour s'extraire et partir librement
\end{enumerate}
Ces deux problèmes ont des analogies classiques. Il s'agit des état lié et état de diffusion.\\

Illustrons. Considérons un potentiel confiné mais de hauteur finie $V_1$. Deux cas sont possibles
\begin{enumerate}
\item[(a)] $E<V_1$ : trajectoire confinée. On observera des états lié donnant lieu à un spectre discret. 
Les niveaux d'énergies seront compris entre $V_{min}<E<V_1$.
\item[(b)] $E>V_1$ : trajectoire libre. On observera des états de diffusion à travers un spectre continu :
continuum d'énergie.
\end{enumerate}
Dans les deux cas, il faudra résoudre l'équation de Schrödinger : la différence se situe dans les CL. \\

\textsc{Cas (a)}\\
Il s'agit d'états stationnaires qui sont des états physiques.
\begin{equation}
\text{État de carrés sommables : }\ \int |\psi(x)|^2\ dx = 1
\end{equation}

\textsc{Cas (b)}\\
Il s'agit d'état stationnaire non physique, car on n'arrive pas à créer un état de 
carré sommable. Interprétation : classiquement la particule est libre, comment 
faire un objet stationnaire ? D'autre part les ondes planes ne peuvent pas 
être normalisées. Cependant, même si ces ondes planes ne sont pas des solutions physiques 
c'est une chouette base (base des états stationnaires non physiques)\\

\subsection{Puits carré infini (1D), théorème de Sturm-Liouville}
	\subsubsection{Puits carré infini (1D)}
Dans ce cas, le potentiel vaut
\begin{equation}
V(x) \left\{\begin{array}{ll}
= 0 &0\leq x\leq L\\
=\infty & \text{sinon}
\end{array}\right.
\end{equation}
Si on résout l'ED de Schrödinger
\begin{equation}
-\frac{\hbar^2}{2m}\frac{\partial^2}{\partial x^2}\psi(x) = E\psi(x)
\end{equation}
En posant le nombre d'onde $k=\frac{\sqrt{2mE}}{\hbar}$
\begin{equation}
-\frac{\partial^2}{\partial x^2}\psi(x) = k^2\psi(x)
\end{equation}
La solution bien connue vaut alors
\begin{equation}
\psi(x) = A\sin(kx) + B\cos(kx)
\end{equation}
En appliquant les C.L. ($\psi(0)=\psi(L)=0$) il en vient que $B=0$ et $\sin(kL)=0$. Cette 
dernière condition fait apparaître la quantification : $k=\dfrac{n\pi}{L},\ n=1,2,\dots$. 
La condition de normalisation nous donne la valeur de $A$ : $\int |\psi(x)|^2\ dx = 1\ \longrightarrow\ 
A = \sqrt{\frac{2}{L}}$.
\begin{equation}
\psi_n(x) =\sqrt{\frac{2}{L}}\sin\left(n\pi\frac{x}{L}\right),\qquad\qquad E_n=\frac{k^2\hbar^2}{2m}=
\frac{n\pi^2\hbar^2}{2mL^2}
\end{equation}
où $n$ est un \textit{nombre quantique}. Il s'agit de la solution d'une corde vibrante. 
		
		\subsubsection{Théorème de Sturm-Liouville}
		On remarque que $n$ correspond aux nombres de nœuds. Le théorème de Sturm Liouville nous dit que 
		cette propriété est totalement générale, ce n'est pas une propriété du puits infini.
		
		\begin{center}
		  \textit{Les niveaux d'énergie 
successifs correspondent à un nombre de nœuds croissants.}
		 \end{center} 
		 Plus on monte en énergie, plus on a de noeuds.\\

		Pour conclure, si on regarde la cas réaliste d'un puits fini, tant qu'on regarde 
		des états dont l'énergie est dans le puits ils sont lié, au dessus ça sera un 
		continuum. Près du bord, on retrouve des ondes évanescentes (décroissance exponentielles). 


	\subsection{Particule dans une boite (3D), cellules de l'espace de phases}
		\subsubsection{Particule dans une boite (3D)}	
		Notre potentiel est une belle boite d'allumettes :
		\begin{equation}
		V^3(x,y,z) = V_1(x) + V_2(y)+V_3(z)\qquad\text{ où }\quad V_i(x_i) \left\{\begin{array}{ll}
		=0 & 0\leq x_1\leq L_i\\
		=\infty & \text{sinon}
		\end{array}\right.
		\end{equation}				
		Écrivons l'équation aux valeurs propres correspondante
		\begin{equation}
		\left[-\frac{\hbar^2}{2m}\left(\dfrac{\partial^2}{\partial x^2}+\dfrac{\partial^2}{\partial y^2}+
		\dfrac{\partial^2}{\partial z^2}\right)+V^3(x,y,z)\right]\psi(x,y,z) = E\psi(x,y,z)
		\end{equation}
		où on utilise la méthode de séparation des variables pour avoir $\psi(x,y,z) = \psi_1(x)\psi_2(y)\psi_3(z)$
		 et $E=E_1+E_2+E_3$. De façon compacte, on peut écrire
		\begin{equation}
		\left[-\frac{\hbar^2}{2m}\dfrac{\partial^2}{\partial x_i^2}+V_i(x_i)\right]\psi_i(x_i) = E_i\psi_i(x_i)
		\end{equation}
		La résolution de ce problème nous donne trois nombres quantiques (positifs) $n_1,n_2$ et $n_3$.
		\begin{equation}
		\psi_{n_1,n_2,n_3} = \sqrt{\dfrac{8}{L_1L_2L_3}}\sin\left(n_1\pi\frac{x}{L_1}\right)
		\sin\left(n_2\pi\frac{y}{L_2}\right)\sin\left(n_3\pi\frac{z}{L_3}\right)
		\end{equation}				
		\begin{equation}
		E_{n_1,n_2,n_3} = \dfrac{\pi^2\hbar^2}{2m}\left(\dfrac{n_1^2}{L_1^2}+\dfrac{n_2^2}{L_2^2}+
		\dfrac{n_3^2}{L_3^2}\right)
		\end{equation}
		On remarque que si $L_1=L_2$ on peut échanger le rôle de $n_1$ et $n_2$, 	cela ne change rien :
		dégénérescences possibles. Comment essayer de "compter" le nombre d'états ?
		
		
		\subsubsection{Cellules de l'espace de phases}
		Nous allons ici tenter de compter les états (caractériser le spectre en imposant des	valeurs de 
		$n$ est trop tendu).\\
		
		Soit $N(E_0)$, le nombre d'états d'énergie $E_{n_1,n_2,n_3} \leq E_0$. Imposer cela revient à dire 
		que
		\begin{equation}
		\dfrac{n_1^2}{L_1^2}+\dfrac{n_2^2}{L_2^2}+	\dfrac{n_3^2}{L_3^2} \leq \dfrac{2mE_0}{\pi^2\hbar^2}
		\end{equation}
		Une façon approximative de répondre à cette question est d'imaginer que l'on se situe dans un 
		espace à trois dimensions où le membre gauche de l'équation ci-dessus serait un rayon élevé au 
		carré. Dans la direction $x$, on aurait un état à $1/L_1, 2/L_1, 3/L_1,\dots$ de même pour les 
		directions $y$ et $z$, donnant lieu à tout un maillage. La question est alors : \textit{combien 
		de points du maillage se trouvent dans la sphère décrite ci-dessus, de rayon $R=\frac{\sqrt{2mE_0}}{
		\pi\hbar}$ ?}\\
				
		Le nombre de points sera donné approximativement par la densité multipliée par le volume. Sachant 
		que la densité est donnée par l'inverse du pas de maillage ($L_1$ pour la direction $x$) :
		\begin{equation}
		\begin{array}{ll}
		N(E_0) &\approx \text{densité }\times{ volume}\\
		&\DS\approx L_1L_2L_3 \times \frac{4}{3}\pi\frac{(2mE_0)^{3/2}}{\pi^3\hbar^3}
		\end{array}
		\end{equation}
		Posons $\DS p_0 = \sqrt{2mE_0}$
		\begin{equation}
		N(E_0) \approx \dfrac{L_1L_2L_3\times\frac{4\pi}{3}p_0^3}{(\hbar/2)^3} \equiv \dfrac{V_{\text{position}}
		\times V_{\text{impulsion}}}{(\hbar/2)^3}
		\end{equation}
		où $V_{\text{position}}$ est le volume dans l'espace des positions et $V_{\text{impulsion}}$ est 
		la sphère des valeurs possibles de l'impulsion si l'énergie est limitée par $E_0$.\\
				
		On voit apparaître la notion d'espace des phases : l'espace dans lequel on aurait pu écrire 
		l'évolution de la particule si elle était à 1D. Le nombre d'état dans un certain volume de 
		cet espace des phases va être donné par
		\begin{equation}
		N \approx \dfrac{\Delta x\Delta y\Delta z\ \ \Delta p_x\Delta p_y\Delta p_z}{\hbar^3}
		\end{equation}
		Une "parcelle" du quadrillage de l'espace des phase est une cellule des phases : peut permettre 
		de compter des fermion (il y a au maximum un fermion par cellule). C'est une limite, le principe
		d'incertitude ne permet pas de faire mieux que cette cellule. Si on travaille avec des fermions, 
		ceux qui n'occupent qu'une cellule et donc que un état.
		
		

	\subsection{Oscillateur harmonique (1D), énergie du point zéro, théorème du viriel}
		\subsubsection{Oscillateur harmonique (1D)}	
		Notre hamiltonien
		\begin{equation}
		H = \frac{p^2}{2m}+\frac{1}{2}m\omega^2x^2
		\end{equation}
		peut s'exprimer en unités réduites
		\begin{equation}
		\left\{\begin{array}{lll}
		x &= a\overline{x} &= \sqrt{\frac{\hbar}{m\omega}}\overline{x}\\
		p &= \frac{\hbar}{a}\overline{p} &= \sqrt{m\omega\hbar}\overline{p}
		\end{array}\right.
		\end{equation}
		où le $\overline{x}$ signifie "sans dimension". L’hamiltonien devient
		\begin{equation}
		H = \frac{m\omega\hbar}{2m}\overline{p}^2+\frac{1}{2}m\omega\frac{\hbar}{m\omega}\overline{x}^2
		= \frac{\hbar\omega}{2}(\overline{p}^2+\overline{x}^2)
		\end{equation}
		Après résolution
		\begin{equation}
		\psi_n(\overline{x}) = \dfrac{1}{\sqrt{\pi^{1/2}2^nn!}}H_n(\overline{x})e^{-\frac{\overline{x}^2}{2}},
		\qquad\qquad E_n = \left(n+\frac{1}{2}\right)\hbar\omega
		\end{equation}
		où $n\geq0$.
		
		
		\subsubsection{Énergie du point zéro}		
		L'énergie à l'origine de notre parabole n'est pas connue car cela impliquerait une connaissance 
		parfaite de la position et de l'impulsion : le principe d'incertitude exclut ce point. Cependant 
		à l'aide de ce principe on peut retrouver l'\textit{énergie du point zéro}, l'énergie la plus 
		basse que possible (ici $\frac{\hbar\omega}{2}$).\\
		
		Pour se faire, calculons la valeur moyenne de $H$
		\begin{equation}
		\langle H\rangle = \frac{1}{2m}\langle p^2\rangle + \frac{1}{2}m\omega^2\langle x^2\rangle = 
		\hbar\omega\left(\langle p^2\rangle+\langle x^2\rangle\right) = \frac{\hbar\omega}{2}\left(\frac{a^2}{
		\hbar^2}\langle p^2\rangle+\frac{1}{a^2}\langle x^2\rangle\right)
		\end{equation}
		La symétrie axiale du potentiel permet d'affirmer que les moyennes des opérateurs
		position et impulsion sont nulles.
		On peut faire apparaître la variance%, la valeur moyenne étant nulle cela ne change rien
		\begin{equation}
		\langle H\rangle \dfrac{\hbar\omega}{2}\left(\frac{a^2}{\hbar^2}\langle\Delta p^2\rangle +\frac{1}
		{a^2}\langle\Delta 		x^2\rangle\right)
		\end{equation}
		En utilisant Heisenberg : $\Delta x^2\Delta y^2 \geq h^4/4$ :
		\begin{equation}
		\langle H\rangle \geq \dfrac{\hbar\omega}{2}\left(\dfrac{a^2}{\hbar^2}\dfrac{\hbar^2}{4\Delta x^2}
		+\frac{1}{a^2}\Delta x^2\right)
		\end{equation}
		En majorant à la grosse louche
		\begin{equation}
		\langle H\rangle \geq \dfrac{\hbar\omega}{2}\left(\frac{a^2}{4\Delta x^2}+\frac{\Delta x^2}{a^2}\right) 
		= \frac{\hbar\omega}{2}\left(\frac{1}{4\Delta \overline{x}^2}+\Delta \overline{x}^2\right)
		\end{equation}
		Calculons le minimum de la dernière parenthèse : $f(t) = \frac{1}{4t}+t \rightarrow f' = -\frac{1}{4t^2}
		+1 =0\Leftrightarrow t = 1/2 \rightarrow f(1/2)=1$. On a donc
		\begin{equation}
		\langle H\rangle \geq\dfrac{\hbar\omega}{2}
		\end{equation}		
		On prouve que l'énergie moyenne doit au moins valoir ça. Si l'énergie moyenne ne peut pas 
		être en dessous de ça, aucune des valeurs ne le peut. Le niveau fondamental va "saturer" le 
		principe d'incertitude (l’inégalité devient une égalité) :
		\begin{equation}
		\left\{\begin{array}{ll}
		\Delta \overline{x}^2 &= 1/2\\
		\Delta \overline{p}^2 &= 1/2		
		\end{array}\right.
		\end{equation}
		Ceci correspond à des états gaussiens. %Ce sont ces états qui saturent le principe.
		C'est une des propriétés des états gaussiens : ils saturent toujours les principes
		d'incertitude, y compris en ce qui concerne les transformées de Fourier.
		 On peut vérifier
		que l'état fondamental est bien gaussien, tout est donc cohérent.
	

		\subsubsection{Théorème du Viriel}
		Ce théorème donne un lien entre l'énergie cinétique moyenne et l'énergie potentielle moyenne. Il 
		a été démontré en mécanique classique. Partons de l'O.H.
		\begin{equation}
		H = \frac{p^2}{2m}+\frac{1}{2}m\omega^2x^2
		\end{equation}
		Calculons le commutateur (même si $\hat{x}\hat{p}$ n'est pas observable, rien ne nous empêche de 
		le calculer (sans les $\hat{\ }$)
		\begin{equation}
		\begin{array}{lll}
		[H,\hat{x}\hat{p}] &= \frac{1}{2m}[p^2,xp] &+ \frac{1}{2}m\omega^2[x^2,xp]\\
		&=\frac{1}{2m}(x[p^2,p]+[p^2,x]p) &+ \frac{1}{2}m\omega^2(x[x^2,p]+[x^2,x]p)\\
		&=\frac{1}{2m}(p[x,p]p + [p,x]p^2) &+ \frac{1}{2}m\omega^2(x^2[x,p]+x[x,p]x)\\
		&=-\frac{1}{2m}2i\hbar p^2 &+ \frac{1}{2}m\omega^2 2i\hbar x^2\\		
		&= -2i\hbar\hat{K}+2i\hbar\hat{V}
		\end{array}
		\end{equation}
		Pour démontrer le théorème, supposons que l'on soit dans un état stationnaire et essayons d'exprimer 
		le théorème d'Ehrenfest
		\begin{equation}
		\dfrac{d}{dt}\langle A\rangle_\psi = \frac{1}{i\hbar}\langle[A,H]\rangle_\psi = 0\qquad \forall A
		\end{equation}
		Ceci étant valable $\forall \hat{A}$, on peut l'appliquer à $\hat{x}\hat{p}$ même si cet opérateur 
		n'est pas un observable. Dans un état stationnaire, la valeur moyenne doit être nulle et donc
		\begin{equation}
		\langle[\hat{x}\hat{p},\hat{H}]\rangle_\psi\qquad\Longrightarrow\qquad \langle \hat{K}\rangle_\psi =
		\langle \hat{V}\rangle_\psi 
		\end{equation}
		où $\psi$ est un état stationnaire. Dans le cas d'un état stationnaire, la moyenne de l'énergie 
		cinétique vaut bien celle de l'énergie potentielle. Dans un cas plus	général	
		\begin{equation}
		V(x) = \lambda x^m
		\end{equation}
		Le théorème du Viriel peut s'écrire
		\begin{equation}
		2\langle\hat{K}\rangle = m\langle\hat{V}\rangle
		\end{equation}
	




